\chapter{Intro to Web}

The web is a collection of data and services. Such services are provided by web servers and accessed through browsers. The web is not the Internet.

\section{URL}
An URL uniquely identifies a piece of data on the web. It consists of several parts:

\medskip
\begin{tabular}{|c|c|c|}
    \hline
    \texttt{https://} &
    \texttt{www.example.com} &
    \texttt{/index.html} \\
    \hline
\end{tabular}

\begin{enumerate}
    \item \emph{Protocol (scheme)} tells the browser how to retrieve the resource.
    \item \emph{Location (domain)} specifies which web server to contact to retrieve the resource.
    \item \emph{Path} indicates which resource on the web server to request.
\end{enumerate}

\section{HTTP}
HTTP is a protocol used to request and retrieve data from a web server. In this model, the browser (client) sends a request to the web server, which processes the request and sends a response back.

\subsection{Structure of a request}
A request consists of a URL path, method, and data. We will focus on two types of request methods:
\begin{itemize}
    \item GET: Intended for requests that don't change server-side state, i.e. "gets" information from the server.
    \item POST: Intended for requests that update server-side state, i.e. "posts" information to the server.
\end{itemize}

\section{Elements of a webpage}
The HTTP protocol returns arbitrary files and the browser interprets the data according to a specified media type. Much of the web is build in three languages: HTML, CSS, and JavaScript.

\subsection{HTML}
HTML allows users to create structured documents with paragraphs, links, embedded images, etc. HTML frames pose a security risk since the inner page may be from a malicious source. To protect against this, modern browsers enforce frame isolation, which ensures that outer and inner pages cannot change the contents of one another.

\subsection{CSS}
CSS lets users modify the appearance of an HTML page. It can be extremely powerful if used maliciously.

\subsection{JavaScript}
JavaScript is a high-level programming language for running code in the web browser. It is client-side, meaning that it runs in the browser not the server. 

\medskip
Although JavaScript is an interpreted language, it is fast. Most web browsers implement JavaScript as a just-in-time compiler, dynamically converting it into machine code to speed up execution.

\medskip
Since the web browser runs JavaScript from external websites, a vulnerability in the browser's JavaScript interpreter/compiler is extremely dangerous. Because this language is so powerful, modern web browsers run JavaScript in a sandbox to prevent the code loaded from a webpage from accessing sensitive data on your computer.

\subsection{DOM}
After the browser receives data from the server, HTML and CSS are pared into a DOM (Document Object Model). The DOM is a cross-platform model for representing and interacting with objects in HTMl. JavaScript is then applied on the DOM and lastly, the browser renders the DOM to display a webpage to the user.

\section{Same-origin policy}
Modern web browsers defend against tampering from malicious websites by adopting the \emph{same-origin policy}. This ensures that two websites with different origins cannot interact with each other.

\medskip
There are a few exceptions to this rule:
\begin{itemize}
    \item JavaScript runs with the origin of the page that loads it.
    \item Websites can fetch and display images from other origins.
\end{itemize}