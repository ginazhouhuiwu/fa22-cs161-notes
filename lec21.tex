\chapter{DNSSEC}

\section{DNS security extensions (DNSSEC)}
We want DNSSEC to defend against malicious name servers, so we choose root servers as the implicit \emph{trust anchor}. We then \emph{delegate trust} from the root to all legitimate name servers through signing public keys. The root's public key is hardcoded.

\medskip
DNSSEC ensures integrity on the results and provides object security. In contrast, DNS over TLS provides channel security.

\section{DNSSEC record types}
To store cryptographic information in DNS messages, we introduce the following record types:
\begin{itemize}
    \item \texttt{DNSKEY} \emph{type record} encodes a name server's own public key.
    \item \texttt{RRSIG} \emph{(resource record signature) type record} is a signature on a set of multiple other records in the message, all of the same type.
    \item \texttt{DS} \emph{(delegated signer) type record} is a hash of the signer’s name and a child’s public key; used to delegate trust.
\end{itemize}

\section{Key signing and zone signing keys}
A key change is necessary if an attacker compromises a private key. This process is complicated as the name server must inform its parent, who then changes its DS record. 

\medskip
To change keys easily, each DNSSEC name server generates two public/private key pairs. The \emph{key signing key (KSK)} is only used to sign the zone signing key, and the \emph{zone signing key (ZSK)} is used to sign everything else.

\section{Nonexistent domains}
Instead of signing individual nonexistent domains, name servers precompute signatures on ranges of nonexistent domains to prove that a domain does not exist beforehand.

\subsection{NSEC}
\emph{NSEC records} prove nonexistence of a record type (NOERROR) and domain (NXDOMAIN). The vulnerability is that each time we query for a nonexistent domain, we discover two valid domains. An attacker can walk through the record and learn the names of every subdomain. This is known as \emph{domain enumeration}.

\subsection{NSEC3}
To solve the issue, \emph{NSEC3} stores pairs of adjacent hashes of every domain name instead of the actual domain name. However, domain enumeration is still possible by brute forcing through all reasonable domain names. This only prevents attackers from learning long, random domain names. The only real way to prevent enumeration is online signature generation with the private key.