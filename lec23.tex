\chapter{Intrusion Detection}
We will first introduce the three broad types of detectors and then discuss four main detection strategies.

\section{Path traversal attack}
The attacker access unauthorized files on a remote server by using '.../' to reference the parent directory and enter other directories. To detect such attack, we must check that user input is not interpreted as a file path.

\section{Network intrusion detection system (NIDS)}
NIDS is installed between the router and the rest of the Internet, so it sees all packets sent to and received from the outside Internet.

\subsection{Advantages of NIDS}
\begin{itemize}
    \item A single detector can cover the entire network, so it is cheap and has low management overhead.
    \item As network gets bigger, computing power to power NIDS scale linearly.
    \item End systems are unaffected as NIDS does not consume any resources on end systems, so it can be used for added security on an existing system.
\end{itemize}

\subsection{Drawbacks of NIDS}
\begin{itemize}
    \item \emph{Inconsistent interpretation}: Inputs are interpreted and parsed differently between the NIDS and the end system. Additionally, what the NIDS sees does not match what the end system sees (e.g. packet's time-to-live (TTL)). This leads to vulnerability to evasion attacks.
    
    To defend against ambiguity, we must impose a canonical/normalized form for all inputs or analyze all possible interpretations.
    
    \item \emph{Encrypted traffic}: TLS is end-to-end secure, so NIDS cannot read encrypted traffic. One possible solution is to give the NIDS access to all the network's private keys.
\end{itemize}

\section{Host-based intrusion detection system (HIDS)}
HIDS is installed directly on the end hosts.

\subsection{Advantages of HIDS}
\begin{itemize}
    \item HIDS have fewer inconsistency issues since it is located on the same machine that is receiving and interpreting the requests.
    
    \item Works for encrypted messages.
    
    \item Can protect against non-network threats, e.g. malicious user inside the network.
    
    \item HIDS performance scales better than NIDS as it does not get overwhelmed.
\end{itemize}

\subsection{Drawbacks of HIDS}
\begin{itemize}
    \item Expensive since a HIDS must be installed for every machine on the network.
    
    \item HIDS don't defend against evasion attacks. In particular, it is vulnerable to path traversal attacks through UNIX file name parsing.
\end{itemize}

\section{Logging}
Logging is analyzing logs generated by web servers to detect intrusion detection.

\subsection{Advantages of logging}
\begin{itemize}
    \item Many modern servers already have built-in logging systems.
    \item Similar to HIDS, logging uses information directly from the end host and avoids many parsing inconsistencies and encrypted traffic issues.
\end{itemize}

\subsection{Drawbacks of logging}
\begin{itemize}
    \item Logging has no real-time detection.
    \item Evasion attacks under Unix file name parsing are still possible.
    \item Attackers can change logs to erase evidence of attack.
\end{itemize}

\section{Detection errors}
Detection accuracy is accessed in terms of:
\begin{itemize}
    \item False positive rate (FPR): The probability that the detector alerts, given there is no attack.
    \item False negative rate (FNR): The probability the detector does not alert, given there is an attack.
\end{itemize}

\subsection{Base rate fallacy}
The base rate (i.e. general prevalence) is often neglected. In situations where there are more false positives than true positive, when a detector with a low false positive rate is detected, it is still highly unlikely that an attack has happened

\subsection{Combining detectors}
\begin{itemize}
    \item Parallel composition: alert if either detector alerts. Intuitively, this combination generates more alerts. This reduces false negative rate but increases false positive rate.
    
    \item Series composition: alert only if both detectors alert. Intuitively, this combination generates fewer alerts. This reduces false positive rate but increases false negative rate.
\end{itemize}

\section{Signature-based detection}
Idea: Look for activity that matches the structure of a known attack (\emph{blacklisting}).

\medskip
Benefits:
\begin{itemize}
    \item Conceptually simple and easy to build up libraries of attacks.
\end{itemize}

Drawbacks:
\begin{itemize}
    \item Won't catch new attacks without a known signature.
    \item Might not catch variants of known attacks.
\end{itemize}

\section{Anomaly-based detection}
Idea: Develop a model of what normal activity looks like. Flag any activity that deviates from normal activity (\emph{whitelisting}).

\medskip
Benefits:
\begin{itemize}
    \item Can catch new attacks.
\end{itemize}

Drawbacks:
\begin{itemize}
    \item Defining normal behavior is difficult. This strategy is extremely impractical. 
\end{itemize}

\section{Specification-based detection}
Idea: Manually specify what normal activity looks like. Flag any activity that deviates from normal activity (\emph{whitelisting}). Look for evidence of compromise.

\medskip
Benefits:
\begin{itemize}
    \item Can detect new attacks.
    \item Low false positive rate if we properly specify all allowed behavior.
\end{itemize}

Drawbacks:
\begin{itemize}
    \item Manually intensive to specify allowed behavior.
    \item Requires constant updates as things change.
\end{itemize}

\section{Behavioral detection}
Idea: Look for evidence of compromise. Unlike the other models, this strategy does not search for attack patterns in the input and instead looks for malicious behavior.

\medskip
Benefits:
\begin{itemize}
    \item Can catch new attacks.
    \item Rarely occurs in non-attack circumstances, so false positive rate is low.
\end{itemize}

Drawbacks:
\begin{itemize}
    \item Attack is only detected after it has started.
\end{itemize}