\chapter{BGP, TCP, and UDP}

\section{Subnets}
IP routes by \emph{subnets}, groups of addresses with a common prefix. By splitting the Internet up into subnets, the process of routing packets across multiple nodes on different LANs becomes is more efficient. 

\medskip
To send a packet to another computer on the same local network, the client first verifies that the recipient's IP address is in the same subnet (same IP prefix) and then uses ARP to translate it into a MAX address.

\medskip
To send a packet to another computer on a different local network, the client first sends the packet to the gateway and the packet is passed into the general Internet. The Internet is a network of networks comprised of many \emph{autonomous systems (ASs)}. Each AS is uniquely identified by its autonomous system number (ASN) and handles its own internal routing.

\section{Border gateway protocol (BGP)}
BGP is the protocol ASs use to communicate. Each router announces what networks it can provide and the AS path that packets would follow. The process continues until the entire Internet is connected into a graph with many paths between ASs.

\subsection{BGP attack}
A lying AS can say that it is responsible for a network it isn't, thus redirecting all traffic to itself. There is not much defense against a malicious AS. Instead, we rely on higher layer protocols to guarantee that messages are sent.

\section{Ports}
Recall that IP is an unreliable best-effort protocol, meaning that packets can be corrupted, reordered, or dropped. IP addresses uniquely identify machines but do not support multiple processes on one machine. The transport layer (layer 4) introduces \emph{port numbers} to distinguish between multiple processes.

\section{Transmission control protocol (TCP)}
TCP is a reliable, in-order, connection-based stream protocol. TCP provides a byte stream abstraction such that low-level packets are hidden from programmers. Streams are broken into \emph{segments}, which contain sequence numbers so the destination can reassmble the stream in order. To keep track of missing packets and guarantee reliability, \emph{acknowledgements (ACKs)} are sent for each sequence number.

\medskip
A unique TCP connection is identified by a 5-tuple of Client IP Address, Client Port, Server IP Address, Server Port, and Protocol (TCP).

\subsection{Three-way TCP handshake}
\begin{enumerate}
    \item Client sends a \emph{SYN packet} to the server and sets an \emph{initial sequence number (ISN)}.
    \item Server sends back a \emph{SYN-ACK packet} and sets the sequence number field to another random number. 
    
    Each TCP connection requires two sets of 32-bit sequence numbers.
    \item Client responds with an \emph{ACK packet}.
\end{enumerate}

To end a connection, one side sends a packet with the \emph{FIN} flag. To abort a connection, a \emph{RST} flag is set.

\subsection{TCP packet injection}
The attacker can spoof a malicious packet and fill in the header to mimic an authentic TCP connection. A related attack is \emph{RST injection}, where the attacker forcibly terminates a connection.

\medskip
TCP by itself provides no confidentiality or integrity guarantees. To prevent injections, we rely on TLS, a higher-layer protocol.

\section{User datagram protocol (UDP)}
UDP is a best-effort transport layer protocol with no reliability or ordering guarantees. Applications break their data into datagrams, which are sent and received as a single unit. UDP is used over TCP when speed is a concern, since it requires no handshake at the start of each connection.
