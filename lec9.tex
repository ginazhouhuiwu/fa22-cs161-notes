\chapter{Public-Key Encryption and Digital Signatures}

\section{Public-key encryption}
In a public-key (asymmetric) cryptosystem, each person has two keys, public and private. Everybody can encrypt using the shared public key but only the intended recipient can decrypt using their private key.

\section{El Gamal encryption}
El Gamal is a modification of Diffie-Hellman to support encrypting and decrypting messages directly.
\begin{itemize}
    \item KeyGen(): Bob generates private key $b$ and public key $B = g^b \mod p$.
    
    Intuition: Bob is completing his half of Diffie-Hellman.
    
    \item Enc(B, M): Alice generates random $r$ and computes $C_1 = R = g^r \mod p$. Alice then computes $C_2 = M \times B^r \mod p$ and sends $C_1, C_2$.
    
    Intuition: Alice is completing her half of Diffie-Hellman, deriving the shared secret, and sending back a multiple of the shared secret.
    
    \item Dec(b, $C_1$, $C_2$): Bob computes $C_2 \times C_1^{-b} = M \times B^r \times R^{-b} = M \times g^{br} \times g^{-br} = M \mod p$.
\end{itemize}

\medskip

El Gamal is not used in practice as it is not secure and the adversary can tamper with the message (i.e. \emph{malleability}).

\section{RSA encryption}
The main idea is we use the public key to encrypt messages and a corresponding private key to decrypt. An attacker cannot break RSA unless they can factor large primes.

\subsection{RSA problem}
Given $N$ and $C=M^e \mod{N}$, it is hard to find $M$. Current best solution is to factor $N$, so the RSA problem is as hard as the factoring problem and therefore secure.

\subsection{Issues of RSA}
RSA is deterministic and not IND-CPA secure.  Sending the same message encrypted with different public keys also leaks information. As a result, we add randomness through a \emph{padding mode}.

\medskip
\emph{Note}: Determinism does NOT imply IND-CPA security! IND-CPA security means that no information is leaked about the plaintext. Simply adding randomness upon RSA does not guarantee security.

\subsection{Optimal asymmetric encryption padding (OAEP)}
OAEP is a padding scheme that generates a random symmetric key, uses the key to scramble the message, and encrypts both the scrambled messange and the random key. 

\section{Hybrid encryption}
Public-key encryption is slow and can only encrypt small messages. In real cryptographic systems, we encrypt data under a randomly generated key $K$ using symmetric encryption and encrypt $K$ using asymmetric encryption. Now we can encrypt large amounts of data quickly using symmetric encryption and still have the security of asymmetric encryption.

\section{Digital signatures}
Digital signatures are the public-key version of MACs, which provide integrity and authenticity to data. In a digital signature scheme, only the owner can generate a signature using the private key (\emph{signing} key) while everyone can verify it with the public key (\emph{verification} key).

\begin{itemize}
    \item KeyGen() $\rightarrow$ PK, SK: Generate the verify key PK and signing key SK.
    \item Sign(SK, M) $\rightarrow$ sig: Sign the message M using signing key SK to produce signature sig.
    \item Verify(PK, M, sig) $\rightarrow$ {0, 1}: Verify the signature sig on message M using the verify key PK and output 1 if valid and 0 if invalid.
\end{itemize}

\section{RSA signatures}
RSA signature is the same as encryption, but we encrypt the hash with the private key.