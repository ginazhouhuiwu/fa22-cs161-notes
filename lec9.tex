\chapter{Public-Key Encryption and Digital Signatures}

\section{Public-key encryption}
In a public-key (asymmetric) cryptosystem, each person has two keys, public and private. Everybody can encrypt using the shared public key but only the intended recipient can decrypt using their private key.

\section{El Gamal encryption}
El Gamal is a modification of Diffie-Hellman to support encrypting and decrypting messages directly.
\begin{itemize}
    \item Key generation: Bob generates private key $b$ and public key $B = g^b \mod p$.
    \item Encryption: Alice generates random $r$ and computes $C_1 = R = g^r \mod p$. Alice then computes $C_2 = M \times B^r \mod p$ and sends $C_1, C_2$.
    \item Decryption: Bob computes $C_2 \times C_1^{-b} = M \times B^r \times R^{-b} = M \times g^{br} \times g^{-br} = M \mod p$.
\end{itemize}

\medskip

El Gamal is not used in practice as it is not secure and the adversary can tamper with the message (i.e. \emph{malleability}).

\section{RSA encryption}
The main idea is we use the public key to encrypt messages and a corresponding private key to decrypt. An attacker cannot break RSA unless they can factor large primes.

\medskip
RSA is deterministic and not IND-CPA secure.  Sending the same message encrypted with different public keys also leaks information. As a result, we add randomness through a \emph{padding mode}.

\subsection{Optimal asymmetric encryption padding (OAEP)}
OAEP is a padding scheme that generates a random symmetric key, uses the key to scramble the message, and encrypts both the scrambled messange and the random key. 

\section{Digital signatures}
Digital signatures are the public-key version of MACs, which provide integrity and authenticity to data. In a digital signature scheme, only the owner can generate a signature using the private key (\emph{signing} key) while everyone can verify it with the public key (\emph{verification} key).

\section{RSA signatures}
RSA signature is the same as encryption, but we encrypt the hash with the private key.