\chapter{Cookies and CSRF}

\section{Cookies}
HTTP is a stateless protocol, meaning that the web server processes each request independently. In order to customize responses (e.g. enable dark mode on a certain website), data must be stored in the browser to maintain state across multiple requests. Such data are \emph{cookies}.

\subsection{Cookie attributes}
Each cookie is a name-value pair. Additional cookie attributes help browser determine which cookies should be attached to each request:
\begin{itemize}
    \item \emph{Domain} and \emph{Path} tell the browser which URLs to send the cookie to.
    \item \emph{Secure} tells the browser to only send the cookie over a secure HTTPS connection.
    \item \emph{HttpOnly} prevents JavaScript from accessing and modifying the cookie.
    \item \emph{Expires} field indicates when to stop remembering the cookie.
\end{itemize}

\subsection{Cookie policy}
When a server sets a cookie, the cookie's domain must be a URL suffix of the server's URL. The only exception is that cookies cannot have domains set to a top-level domain (e.g. .com or .edu).

\medskip
\emph{Note}: Cookie policy is not the same as same-origin policy.

\section{Session authentication}
When a user sends a login request with a valid username and password, the server generates a new \emph{session token} and sends it to the user as a cookie. In future requests, the browser looks up the corresponding user and attaches the session token cookie.

\section{Cross-site request forgery (CSRF)}
In a CSRF attack, the attacker forces the victim to make an unintended request by exploiting cookie-based authentication.

\subsection{CSRF attack steps}
\begin{enumerate}
    \item User authenticates to the server and receives a cookie with a valid session token.
    \item Attacker tricks the victim into making a malicious request to the server.
    \item The server accepts the malicious request from the victim.
\end{enumerate}

\subsection{CSRF tokens}
A defense against CSRF attacks is including a CSRF token, which is a random secret value provided by the server to the user. The user must attach the same value in the request for the server to accept the request.

\subsection{Referer validation}
In a CSRF attack, the victim usually makes the malicious request from a different website. Thus, another defense against CSRF tokens is to check the Referer field in the HTTP header in a CSRF attack, the victim usually makes the malicious request from a different website. The Referer header indicates which URL the request was made from.