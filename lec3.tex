\chapter{Memory Safety Vulnerabilities}

\section{Buffer overflow vulnerabilities}
C does not have automatic bounds-checking for array or pointer accesses. Attackers exploit the lack of boundaries to control memory that they are not supposed to access.

\medskip

\emph{Example}:

\begin{minted}{c}
    char buf[8];
    int (*fnptr)();
    
    void vulnerable() {
        gets(buf);
    }
\end{minted}

\texttt{fnptr} is a function pointer stored directly above \texttt{buf} in memory. The attacker can overwrite \texttt{fnptr} with any address of their choosing and redirect execution of the program. This enables a \emph{malicious code injection} attack.

\section{Stack smashing}
Attackers can exploit buffer overruns by overwriting saved registers on the stack.

\medskip

\emph{Example}:

\begin{minted}{c}
    void vulnerable() {
        char buf[8];
        gets(buf);
    }
\end{minted}

\emph{Note}: User input is written from lower to higher addresses.

\medskip

When \texttt{vulnerable()} is called, a stack frame is pushed onto the stack. If the input is too long, the code will write past the end of \texttt{buf}, overwrite the SFP, and overwrite the RIP.

\medskip

Suppose the shellcode we want to inject is 8 bytes long. We can enter the following input: \texttt{[shellcode] + [4 bytes of garbage] + [address of buf]}. The garbage bytes overwrite the SFP and we overwrite the RIP with the address of shellcode.

\section{Integer conversion vulnerabilities}

\subsection{Signed/unsigned vulnerabilities}

\emph{Example}:

\begin{minted}{c}
    void vulnerable(int len, chat *data) {
        char buf[8];
        if (len > 8)
            return;
        memcpy(buf, data, len);
    }
\end{minted}

The attacker provides \texttt{len} with a negative value and \texttt{memcpy()} is executed with a negative third argument. C will cast the negative value to an unsigned integer (e.g. \texttt{0xffffffff}) and copy a huge amount of memory into \texttt{buf}, causing buffer overflow.

\subsection{Integer overflow vulnerabilities}

\emph{Example}:

\begin{minted}{c}
    void vulnerable(int len, chat *data) {
        char *buf = malloc(len + 2);
        if (!buf)
            return;
        memcpy(buf, data, len);
    }
\end{minted}

\texttt{len+2} may wrap around if \texttt{len} is too large. For instance, if \texttt{len = 0xffffffff}, then \texttt{len+2} is 1 and the code allocates only 1 byte for the buffer.

\section{Format string vulnerabilities}
\texttt{printf()} takes in a variable number of arguments. When there is a mismatch between the number of format string modifiers and the number of arguments, \texttt{printf()} fetches data from the stack according to the amount of modifiers. The attacker takes advantage of format string vulnerabilities to learn any value stored in memory.

\section{Heap vulnerabilities}

\subsection{Heap overflow}
Objects are allocated in the heap. The attacker overflows the buffer and overwrites the vtable pointer of the next object to point to a malicious vtable with pointers to shellcode.

\subsection{Use-after-free}
When an object is deallocated too early, the attacker allocates memory and overwrites a vtable pointer to point to a malicious vtable. Shellcode is accessed when the deallocated object's function is called.

\section{Serialization}
Serialization libraries can load and save arbitrary objects. The attacker provides a malicious file to be deserialized.

\section{Putting together an attack}
\begin{enumerate}
    \item Find a memory safety vulnerability.
    \item Write malicious shellcode at a known memory address.
    \item Overwrite the RIP with the address of shellcode.
    \item Return from function.
    \item Execute shellcode.
\end{enumerate}

\section{Writing robust exploits}
A working exploit is often machine-dependent, thus writing robust exploits is extremely difficult.

\subsection{NOP sleds}
Instead of jumping to an exact address, we can chain a long sequence of NOPs such that landing anywhere in the proximity will bring you to the shellcode.