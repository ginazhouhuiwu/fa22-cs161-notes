\chapter{Malware and Worms}

\section{Malware}
\emph{Malicious software} is attacker code running on victim computers. It propogates through spreading copies. \emph{Self-replicating code} is a code snippet that outputs a copy of itself and can be used to automatically propagate malware.

\section{Viruses}
\emph{Viruses} are code that requires user action to propagate, usually infecting a computer by altering stored code and spread when the user runs the code.

\subsection{Propagation strategies}
\begin{itemize}
    \item Infect existing code that user needs to execute (e.g. opening an app, system startup).
    \item Modify existing code to include malcode.
    \item Infect more systems when malcode runs.
\end{itemize}

\subsection{Detection strategies}
\emph{Signature-based detection} systems use patterns of known attacks, create a signature on the virus, and capture virus by looking for bytes corresponding to the malcode on other systems. Antivirus software includes a checklist of common viruses.

\subsection{Arms race}
There is an active arms race between attackers writing viruses and antivirus companies detecting them. Attackers seek out \emph{evasion} strategies by changing the appearance of a virus so each copy looks different, in turn making signature-based detection harder.

\section{Polymorphic code}
\emph{Polymorphic code}: Each time the virus propagates, it inserts an encrypted copy of the code. The malcode also includes the key and decryptor. The goal is not confidentiality, but rather \emph{obfuscation} such that the virus looks different.

\subsection{Polymorphic code defenses}
\begin{itemize}
    \item Add a signature for detecting the decryptor code.

    Issues: Less code to match against produces more false positives; decryptor code could be scattered across memory space.

    \item Run potential malcode in a sandbox.

    Issues: Legitimate programs might perform similar operations.
\end{itemize}

\section{Metamorphic code}
\emph{Metamorphic code}: Each time the virus propagates, it generates a semantically different version of the code. It includes a code rewriter to create minor differences in execution between copies.

\subsection{Metamorphic code defenses}
\begin{itemize}
    \item Behavior detection: Analyze effects of the instructions rather than syntax.
    
    Issues: Viruses can detect that the code is being analyzed and choose different behavior or delay analysis.

    \item Flag unfamiliar code: Treat unseen code as more suspicious.
\end{itemize}

\section{Worms}
\emph{Worms} propagate without user action, usually infecting a computer by altering some already-running code.

\subsection{Propagation strategies}
Worms randomly choose machines by randomly generating 32-bit IP addresses and connect to them (or use a pre-generated "hit-list"). They can spread very quickly via parallelizing the process of propagation and replication (\emph{logistic} growth like infectious diseases). Viruses have the same property but spread more slowly since user action is needed.

\section{Rootkits}
\emph{Rootkits} are malcode in the operating system that hides its presence. In its presence, users cannot trust the disk or OS.